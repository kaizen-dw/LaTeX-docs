\documentclass{article}
\usepackage{amsmath, amssymb, amsfonts}
\usepackage{float}
\parindent 0px
\pagestyle{empty}

\title{LaTeX Tutorial}
\author{Dale Walter G. Hicban}

\begin{document}
\maketitle

\break

\section{COMMON MATHEMATICAL NOTATION}

superscripts $$2x^3$$
$$2x^[34]$$
$$2x^[  3x+4  ]$$
$$2x^[  3x^4+5  ]$$

subscripts
$$x_1$$
$$x_{12}$$
$$x_{1_2}$$
$$x_{  1_{2_3}  }$$
$$a_0, a_1, a_2, \ldots, a_{100}$$

Greek letters
$$\pi$$
$$\Pi$$
$$\alpha$$
$$A=\pi r^2$$

Trig functions
$$y=\sin x$$
$$y=\cos x$$
$$y=\csc \theta$$
$$y=\sin^{-1} x$$
$$y=\arcsin x$$

Log functions
$$y=\log x$$
$$y=\log_5 x$$
$$y=\ln x$$

Roots
$$\sqrt{2}$$
$$\sqrt[3]{2}$$
$$\sqrt{  x^2+y^2  }$$
$$\sqrt{ 1+\sqrt{x} }$$

Fractions
$$\frac{2}{3}$$
About $\frac{2}{3}$ of the glass is full.\\[6pt]
About $\displaystyle \frac{2}{3}$ of the glass is full.\\[6pt]
About $\dfrac{2}{3}$ of the glass is full.\\[6pt]

$$\frac{  \sqrt{x+1}  }{  \sqrt{x+2}  }$$
$$\frac{1}{  1+\frac{1}{x}  }$$\\[12pt]

\section{BRACKETS, TABLES, ARRAYS}
	The distributive property states that $a(b+c)=ab+ac$, for all $a, b, c \in \mathbb(R)$.\\[6pt]
	The equivalent class of $a$ is $[a]$.\\[6pt]
	The set $A$ is defined to be $\{1, 2, 3\}$.\\[6pt]
	The movie ticket costs $\$11.50$.

$$2\left(  \frac{1}{x^2-1}  \right)$$
$$2\left[  \frac{1}{x^2-1}  \right]$$
$$2\left\{  \frac{1}{x^2-1}  \right\}$$
$$2\left \langle \frac{1}{x^2-1}\right \rangle $$
$$2\left | \frac{1}{x^2-2}\right | $$

$$\left.\frac{dy}{dx}\right|_{x=1}$$

$$\left(  \frac{1}{1 + \left( \frac{1}{1+x} \right)}  \right)$$


\end{document}